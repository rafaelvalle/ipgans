\begin{hyp}[H\ref{hyp:generate}] \label{hyp:generate}
Generative models can approximate the distribution of real data and hallucinate
fake data that resembles real data and has some variety. 
\end{hyp}

Although this hypothesis is trivial for experiments that have already been
conducted, it is the first condition for our experiments with polyphnic music
and speech data. To our knowledge there are no publications where GANs are 
successful in hallucinating polyphonic music and speech data. During out 
experiments we prove that these hypotheses hold.

\begin{hyp}[H\ref{hyp:features}] \label{hyp:features}
The real data has useful properties that can be extracted computationally.
\end{hyp}
By useful we refer to properties that are closely related to the real data
itself. For example, computing the distribution MNIST pixel values might be not
useful for assessing drawing quality. However, it might be useful to evaluate
if a random MNIST samples is real or fake data.

\begin{hyp}[H\ref{hyp:visual}] \label{hyp:visual}
The fake data has properties that are hardly noticed with visual inspection of
samples.
\end{hyp}
Visual inspection of generated samples has become the norm for the evaluation of
samples generated using the GAN framework. We investigate if there are
properties common to all GAN samples or properties that significantly differ
between the real data and the fake data. 
This hypothesis supports the next hypothesis related to adversarial attacks. 

\begin{hyp}[H\ref{hyp:difference}] \label{hyp:difference}
The difference in properties can be used to identify the source (real or fake)
\end{hyp}
The development of generative models foreshadow the iminent
rise of adversarial attacks. We investigate if these differences can be used to
detect the source of the data (real, GAN or adversarial attack). 

We call the reader's attention that approximating the distribution over features
computed on the real data does not guarantee that the real data is being
approximated. Formally speaking:
consider $X \sim Z$, i.e. X distributed as Z, and $f(X) \sim W$, where $f: X
\mapsto Y$.
If $A \sim B$ and $B$ approximates $Z$, then $f(A) \sim D$ must also approximate $W$.
However, a distribution that approximates $W$ is not guaranteed to approximate
$Z$.


