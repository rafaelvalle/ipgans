\documentclass{article}

% if you need to pass options to natbib, use, e.g.:
% \PassOptionsToPackage{numbers, compress}{natbib}
% before loading nips_2017
%
% to avoid loading the natbib package, add option nonatbib:
% \usepackage[nonatbib]{nips_2017}

\usepackage{nips_2017}

% to compile a camera-ready version, add the [final] option, e.g.:
% \usepackage[final]{nips_2017}

\usepackage[utf8]{inputenc} % allow utf-8 input
\usepackage[T1]{fontenc}    % use 8-bit T1 fonts
\usepackage{hyperref}       % hyperlinks
\usepackage{url}            % simple URL typesetting
\usepackage{booktabs}       % professional-quality tables
\usepackage{amsfonts}       % blackboard math symbols
\usepackage{nicefrac}       % compact symbols for 1/2, etc.
\usepackage{microtype}      % microtypography
% my imports
\usepackage{ntheorem}       % theorem writing

\title{Quantitative Analysis of GAN samples}

% The \author macro works with any number of authors. There are two
% commands used to separate the names and addresses of multiple
% authors: \And and \AND.
%
% Using \And between authors leaves it to LaTeX to determine where to
% break the lines. Using \AND forces a line break at that point. So,
% if LaTeX puts 3 of 4 authors names on the first line, and the last
% on the second line, try using \AND instead of \And before the third
% author name.

\author{
  Rafael Valle \\
  Center for New Music and Audio Technologies \\
  UC Berkeley \\
  Berkeley, 94709 \\
  \texttt{rafaelvalle@berkeley.edu} \\
  \And
  Wilson Cai\\
  UC Berkeley\\
  \texttt{wcai@berkeley.edu} \\
  \And
  Anish Doshi\\
  UC Berkeley\\
  \texttt{apdoshi@berkeley.edu} \\
  %% \And
  %% Coauthor \\
  %% Affiliation \\
  %% Address \\
  %% \texttt{email} \\
  %% \And
  %% Coauthor \\
  %% Affiliation \\
  %% Address \\
  %% \texttt{email} \\
}

\begin{document}
% \nipsfinalcopy is no longer used

\maketitle

\begin{abstract}
    In this paper we quantitatively compare samples produced with adversarial
    methods, specially Generative Adversarial Networks, and the real data
    distribution. We show that one can produce a useful distance measure
    between real and fake distributions by using the joint probability of 
    marginalized perceptually significant features computed over the real and
    fake data. We provide results on image, music and speech data and show 
    that GAN generated samples have signatures that can be used to detect 
    adversarial attacks.
\end{abstract}

% setup theorems 
\theoremseparator{:}
\newtheorem{hyp}{Hypothesis}

\section{Introduction} \label{sec:introduction}
Since the groundbreaking Generative Adversarial Networks
paper~\cite{goodfellow2014generative} in 2014, most GAN related
publications use a grid of image samples to accompany theoretical and empirical
results. Given this context, the expansion of GAN research to other domains including language 
models~\cite{gulrajani2017improved} and music~\cite{yang2017midigan} display the
need of sample inspection.

Unlike Variational Autoencoders (VAEs) and other
models~\cite{goodfellow2014generative}, most of the evaluation of the output
of Generators trained with the GAN framework is qualitative: authors normally 
list higher sample quality as one of the advantages of their method over other
methods.  Interestingly, little is mentioned about the numerical properties of
GAN samples and how these properties compare to real samples.

In the context of Verified Artificial Intelligence\cite{seshia2016vai}, 
it is hard to systematically verify the Generator and the samples it produces
because verification might depend on the existence of perceptually meaningful features. For example, consider the 
generation of images of humans: although it is possible to compare color histograms 
of real and fake\footnote{Generated samples} samples, we do not yet have robust 
algorithms able to verify if an image follows specifications derived from anatomy. 

This paper is related to this systematic sample verification and focuses on
understanding the numerical properties of GAN samples. We investigate how the
Generator approximates modes 
in the real distribution and verify if the generated samples violate specifications 
derived from the real distribution. We offer the following contributions in this paper: 
\begin{itemize}
\item We show that GAN samples have universal signatures.
\item We show how GAN samples approximate modes of the real distribution.
\item We show significant differences between the marginal distribution of features. 
\item We show GAN samples that violate specifications in the real data.
\end{itemize}

%
\section{Related work}\label{sec:related_work}
Despite its youth, several publications (\cite{arjovsky2017towards}, 
\cite{salimans2016improved}, \cite{zhao2016energy}, 
\cite{radford2015unsupervised}) have investigated the use of the
GAN framework for generation of samples and unsupervised feature learning. 
Following the procedure described in~\cite{breuleux2011quickly} and
used in~\cite{goodfellow2014generative}, earlier GAN papers evaluated
the quality of the Generator by fitting a Gaussian Parzen window\footnote{Kernel
Density Estimation} to the GAN samples and reporting the log-likelihood of the
test set under this distribution. It is known that this method has some drawbacks, 
including its high variance and bad performance in high dimensional
spaces~\cite{goodfellow2014generative}.

Unlike other optimization problems, where analysis of
the empirical risk is a strong indicator of progress, in GANs the decrease in loss 
is not always correlated with increase in image quality~\cite{arjovsky2017wasserstein}, and thus authors still rely on visual 
inspection of generated images. Based on visual inspection, authors confirm that
they have not observed mode collapse or that their framework is robust to mode
collapse if some criteria is met (\cite{arjovsky2017wasserstein}, 
\cite{gulrajani2017improved}, \cite{mao2016least}, \cite{radford2015unsupervised}).
In practice, github issues where practitioners report mode collapse or not enough 
variety abound.

In their brilliant publications, \cite{mao2016least},
\cite{arjovsky2017wasserstein} and \cite{gulrajani2017improved} propose alternative
objective functions and algorithms that circumvent problems that are common when using the
original GAN objective described in~\cite{goodfellow2014generative}. The problems addressed include instability of learning,
mode collapse and meaningful loss curves~\cite{salimans2016improved}.

These alternatives do not eliminate the need or excitement\footnote{Despite of
authors promising on twitter to never train GANs again.} 
of visually inspecting GAN samples during training, nor do they provide
quantitative information about the generated samples. In the following sections, we
will analyze GAN samples and reveal some interesting properties therein. 
In addition to comparing the marginal distribution
of features from the real and fake data, we approach these distributions from
the real data as specifications that can be used to validate the output of GAN Samples. 
We start by enumerating the hypotheses evaluated in this paper.

%In~\cite{berthelot2017began}, the authors propose a solution to the diversity
%problem by introducing a new hyper-parameter $\gamma$ with a loss derived from
%the Wasserstein distance. 

%Naturally, this new hyper-parameter does not target the diverstiy of a specific 
%attribute of the images and the results in the paper suggest that in their experiments 
%$\gamma$ is also correlated with the variety of the color pallete.  
%Theoretically, this is dissonant with research evaluating mode collapse and variety in samples generated with the
%GAN framework.  

%Related to constrained paper, work by Deepak shows a very interesting approach, where summary 
%statistics of the output label are used to train the Generator and evaluate its output. 
%In his paper, Deepak proposes a method that uses a novel loss function to
%optimize for any set of linear constraints on the output space of a CNN.
%DESCRIBE IT MORE.

%Our paper draws inspiration from formal methods and specification mining. 
%We approach such constraints as specifications that are mined from features
%computed over the real data. In addition to comparing the marginal distribution
%of features from the real and fake data, we approach these distributions as
%specifications that can be used to validate the output of GAN Samples. In this
%paper we focus on image representations of numbers, speech and music, including MNIST images, 
%mel spectrograms and piano rolls.

%
\section{Method}\label{sec:method}
In this section we describe our analysis method in detail, including
briefly describing the datasets and features computed, as well as distance
or divergence measures. \subsection{Datasets}
In our experiments we use the MNIST dataset, a MIDI dataset of 389 Bach Chorales downloaded from the web and a subsample of the NIST 2004 telefone
conversational speech dataset with 100 speakers, multiple languages and
on average of 5 minutes per speaker.

\subsection{Property extraction}
The properties extracted from the datasets used on this paper can be
perceptually meaningful or not. We claim that both properties can be used to numerically identify the source of the sample. In the context of this
paper, samples are images of fixed size. Consider the single channel image $I$ with dimensions $R$ by $C$, where $I_{r, c}$ is the pixel intensity of the pixel at row $r$ and column $c$

\subsubsection{Spectral Moments}
The spectral centroid~\cite{peeters2004large} is a feature commonly used in the
audio domain, where it represents the barycenter of the spectrum. This feature
can be applied to other domains and we invite the reader to visualize 
Figure~\ref{fig:centroids} for examples on MNIST and
Mel-Spectrograms~\cite{peeters2004large}. For each column in an image, we 
transform the pixel values into row probabilities by normalizing them by the
column sum, after which we take the expected row value. 
%Given one image column, we define $r$ as the pixel intensity at row $r$, and% 

%\begin{equation}
%    p(r) = \frac{r}{\sum_{r \in R}r}
%\end{equation}

Figure~\ref{fig:mnist_centroids} shows the spectral centroid computed
on sample of MNIST training data.

\begin{figure}[!h]
    \centering
    \begin{subfigure}[b]{0.4\textwidth}
        \includegraphics[width=\linewidth]{mnist_centroids.png}
        \caption{MNIST samples and centroids}
        \label{fig:mnist_centroids}
    \end{subfigure}
    \quad
    \begin{subfigure}[b]{0.4\textwidth}
        \includegraphics[width=\linewidth]{speech_spectral_centroids.png}
        \caption{Mel-Spectrograms and centroids}
        \label{fig:spectrogram_centroids}
    \end{subfigure}
    \caption{Spectral centroids on digits and Mel-Spectrograms}
    \label{fig:centroids}
\end{figure}

\subsubsection{Spectral Slope}
The spectral slope is computed by applying linear regression using a overlapping
sliding window of size 7. Figure~\ref{fig:slopes} shows these
features computed on MNIST and Mel-Spectrograms. 

\begin{figure}[!h]
    \centering
    \begin{subfigure}[b]{0.4\textwidth}
        \includegraphics[width=\linewidth]{mnist_slopes.png}
        \caption{MNIST samples and slopes}
        \label{fig:mnist_slopes}
    \end{subfigure}
    \quad
    \begin{subfigure}[b]{0.4\textwidth}
        \includegraphics[width=\linewidth]{speech_spectral_slopes.png}
        \caption{Mel-Spectrograms and slopes}
        \label{fig:spectrogram_slopes}
    \end{subfigure}
    \caption{Spectral slopes on digits and Mel-Spectrograms}
    \label{fig:slopes}
\end{figure}

\subsection{Generative Models}
We investigate samples produced with the DCGAN architecture using the
Least-Squares GAN (LSGAN)~\cite{mao2016least} and the improved Wasserstein
GAN (IWGAN)~\cite{gulrajani2017improved}. We also compare adversarial MNIST
samples produced with the fast gradient sign method
(FGSM)~\cite{goodfellow2014explaining}.

%
\section{Experiments}\label{sec:experiments}
\subsection{MNIST}
\subsection{Bach Chorales}
\subsection{Speech}

%
\section{Conclusions}\label{sec:conclusions}
In this paper we investigated numerical properties of samples produced 
with adversarial methods, specially Generative Adversarial Networks. We showed
that GAN samples have universal signatures that are dependent on the choice of
non-linearity on the last layer of the generator. In addition, we showed that
adversarial examples produced with the FSGM have properties that can be used to
identify an adversarial attack. Following, we showed that GAN samples smoothly
approximate the dominating modes of the distribution and that this information
can be used to identify the source of the data. Last, we showed that samples
generated with GANs violate specifications and do not provide guarantees on 
satisfaction of simple specifications. With this we hope to call attention to 
the necessity of the development of verified AI and better understanding of GAN
generated samples.

%

\subsubsection*{Acknowledgments}
\input{acknowledgments}

\section*{References}
\bibliography{seq_gan}

\end{document}
